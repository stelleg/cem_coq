\documentclass[preprint]{sigplanconf}

\usepackage{tabularx}
\usepackage{amsfonts}
\usepackage{amsmath}
\usepackage{natbib}
\usepackage{graphicx}
\usepackage{tikz}
\usepackage{mathtools}
\usetikzlibrary{chains,fit,shapes,calc}
\usepackage{verbatim}
\usepackage{semantic}
%%\usepackage{DejaVuSansMono}
%%\usepackage{fontspec}
\usepackage{tabu}
\usepackage{amsthm}
\usepackage{mathptmx}
\usepackage{todonotes}
\usepackage{listings, lstcoq}
\usepackage{ucs}
\usepackage[utf8x]{inputenc}
\lstset{language=Coq,
        inputpath=code
       }
\newcommand{\concat}{\ensuremath{+\!\!\!\!+\,}}  

\newtheorem{theorem}{Theorem}[section]
\newtheorem{corollary}{Corollary}[theorem]
\newtheorem{lemma}[theorem]{Lemma}

\begin{document}

\conferenceinfo{ICFP 2018}{} 
\copyrightyear{2018} 
\copyrightdata{[to be supplied]} 

\titlebanner{Preprint}        % These are ignored unless

\title{Verifiably Lazy}
\subtitle{Verified Compilation of Call-by-Need}

\authorinfo{George Stelle}
           {University of New Mexico}
           {stelleg@cs.unm.edu}
\authorinfo{Darko Stefanovic}
           {University of New Mexico}
           {darko@cs.unm.edu}
\maketitle

\begin{abstract}
We present a verified compiler that preserves call-by-need semantics when
compiling lambda calculus into a simple assembly language. We use a recently
developed abstract machine that implements call-by-need semantics using shared
environments to implement and reason about the compiler. We show that the
abstract machine ensures that the compiled assembly code implements a
call-by-need natural semantics. In addition to a proof that the memoization of
results is correct, the bisimulation gives us tools for reasoning about time and
space requirements of the machine code. Because call-by-need is an optimization
for call-by-name, we also show the $\beta$ reduction semantics of call-by-name
are preserved.
\end{abstract}

\section{Introduction}
Compilers are an attractive target for verification: the amortized return on
investment is high. Every time a program compiled with a verified compiler is
run, the proof ensures that the semantics of the source program are being
preserved through execution. 

Existing work has focused on verifying \emph{strict}
languages~\cite{chlipala2007certified, leroy2012compcert}. This paper presents
the first machine-verified compiler of a non-strict language. Non-strict languages evaluate bound variables on-demand, instead of precomputing their
value. It is generally accepted that the mapping of a non-strict language to
hardware is harder to reason about.  Indeed, the vast majority of languages
have strict semantics by default largely for this reason. Reasoning formally
about the correctness of a non-strict compiler is similarly difficult. We make
the challenge even greater by ensuring that the most important optimization for
non-strict languages, sharing evaluation results between instance of a
variable, or call-by-need semantics, is implemented correctly. This turns out
to be particularly challenging: one must reason about updating expressions with
values in a heap. 

Our approach is enabled by a recently developed abstract machine, the Cactus
Environment Machine $\mathcal{CE}$ \cite{?}. $\mathcal{CE}$ uses a shared
environment to share results between instances of a variable. It can be compiled
to machine code very succinctly, reducing the load for formal reasoning about
the compiler greatly. It is likely we would not have succeeded in (or even
attempted) creating a verified compiler of call-by-need without this technique.

The compiler and proofs use a number of different representations, including
standard lambda calculus, lambda calculus with deBruijn terms, and locally
nameless representation. The paper proceeds by showing each transformation that
the compiler makes, along with the logical relations between the semantics

\subsection{Main Result}
Here we give a high level overview of the main result of the paper.

Our source language is lambda calculus: 
$$ t ::= x \; | \; \lambda x.t \; | \; t \; t $$

Application is left associative, as usual, and we will generally use natural
numbers for variables. Our target language is a simple machine assembly
language:

\begin{align}
  \tag{Word}   n, p &\in \mathbb{N} \\
  \tag{Registers} r &::= ip \; | \; ep \; | \; r1 \; | \; r2 \; | \; r3 \\
  \tag{Stack}     s &::= [p] \\
  \tag{Write Operands}  wo &::= r \; | \; p \\
  \tag{Read Operands}  ro &::= wo \; | \; n \\
  \tag{Instructions} i &::= \texttt{mov} \; ro \; wo \; 
                       | \; \texttt{jmp} \; ro \; 
                       | \; \texttt{inc} \; wo \;
                       | \; \texttt{dec} \; wo \;
                       | \; \texttt{new} \;  
                       | \; \texttt{push} \; ro \\
                       | \; \texttt{pop} \; wo \\
  \tag{Program}   p &::= [i]
\end{align}

Our machine words are natural numbers, which can be written into registers or
the hea. which is a partial function, or finite map, from pointers $p$ to
words. Our instruction set is simple as well. Given our compiler, which we will
describe in later sections, which compiles a lambda term into a program, we
prove the following main result:

\begin{theorem}[Compiler Correctness]
Call-by-need semantics bisimulate machine semantics, and compilation
preserves this bisimulation relation.
\end{theorem}

We'll formalize this more thoroughly later, but the bisimulation ensures that
we're correctly sharing the results of evaluation down to machine code. Because
the semantics are deterministic, we get the following corollary:   

\begin{corollary}[Correct Results]
If a term $t$ compiles to $p$, then call-by-need evaluates a term $t$ to a value
$v$ \emph{iff} $p$ executes on the machine to a state $s$, where $v$ and $s$ are
related by the bisimulation relation.
\end{corollary}

This corollary says that we get the correct value when we execute the assembly
program on the machine. This is similar, though slightly stronger, to the result
rom \cite{chlipala2007certified}. Chlipala shows the first half of the
\emph{iff}, though the second half is implied by the fact that he's working with
a total language.

\subsection{Outline}
We structure the paper as follows: for each step in the compiler, there are at
least one 

\section{Background and Motivation} \label{sec:back}

This section provides relevant background for the $\mathcal{CE}$ machine,
outlining lambda calculus, evaluation strategies, and Curien's calculus of
closures.

\subsection{Preliminaries}

We begin with the simple lambda calculus ~\cite{barendregt1984lambda}:  $$ t::= x
\; | \;  \lambda x.t \; | \;  t \; t $$ where $x$ is a variable, $\lambda x.t$
is an abstraction, and $t \; t$ is an application. We also use lambda calculus
with deBruijn indices, which replace variables with a natural number indexing
into the binding lambdas.  This calculus is given by the syntax: $$ t::= i \; |
\; \lambda t \; | \; t \; t $$ where $i \in \mathbb{N}$. In both cases, we use
the standard Barendregt syntax conventions, namely that applications are left
associative and the bodies of abstractions extend as far as possible to the
right ~\cite{barendregt1984lambda}.  A \emph{value} in lambda calculus refers to
an abstraction. We are concerned only with evaluation to weak head normal form
(WHNF), which terminates on an abstraction without entering its body.

In mechanical evaluation of expressions, it would be too inefficient to perform
explicit substitution. To solve this, the standard approach uses closures
~\cite{landin1964mechanical,curien1991abstract,jonesstg,biernacka2007concrete}.
Closures combine a term with an environment, which binds the free variables in
the term to closures. 

For a formal basis for closures, we use Curien's calculus of
closures~\cite{curien1991abstract}, given in Figure~\ref{fig:calcclos}.  It is a
formalization of closures with an environment represented as a list of closures,
indexed by deBruijn indices. We will occasionally modify this calculus by
replacing the deBruijn indices with variables for readability, in which case
variables are looked up in the environment instead of indexed, e.g., $t[x = c, y
= c'])$ ~\cite{barendregt1984lambda}. We also add superscript and subscript
markers to denote unique syntax elements, e.g., $t', t_1 \in \textnormal{Term}$. 

\subsection{Evaluation Strategies} \label{sec:eval}

There are three standard evaluation strategies for lambda calculus:
call-by-value, call-by-need, and call-by-name.  Call-by-value evaluates every argument
to a value, whereas call-by-need and call-by-name only evaluate an argument if
it is needed.  If an argument is needed more than once, call-by-name re-computes
the value, where call-by-need memoizes the value, so it is computed at most once.
Thus, call-by-need attempts to embody the best of both worlds---never repeat
work (call-by-value), and never perform unnecessary work (call-by-name). These
are intuitively \emph{good} properties to have, and we illustrate the
correctness of such an intuition with the following example, modified from
~\cite{danvy2013synthetic}:

$$ \overbrace{c_m (c_m (\cdots(c_m}^{m} \; id \; \overbrace{id)\cdots) id)}^{m} \; true \; id
\; bottom $$ where $c_n = \lambda s.\lambda z.\overbrace{s \; (s \cdots (s}^{n}
\; z) \cdots) $, $true = \lambda t.\lambda f.t$, $id=\lambda x.x$, and $bottom =
(\lambda x.x \; x) \lambda x.x \; x$. Call-by-value never terminates,
call-by-name takes exponential time, and call-by-need takes only polynomial time
~\cite{danvy2013synthetic}. Of course, this is a contrived example, but it
illustrates desirable properties of call-by-need.

In practice, however there are significant issues with call-by-need evaluation.
We focus on the following: \emph{Delaying a computation is slower than
performing it immediately.} This issue is well known
\cite{johnsson1984efficient,jonesstg}, and has become part of the motivation
for \emph{strictness analysis}
\cite{mycroft1982abstract,wadler1987projections}, which transforms non-strict
evaluation to strict when possible.

\subsection{Existing Call-by-Need Machines}

Diehl et al. ~\cite{diehl2000abstract} review the call-by-need
literature in detail.  Here we summarize the most relevant points.

The best known machine for lazy evaluation is the Spineless Tagless
G-Machine (STG machine), which underlies the Glasgow Haskell Compiler (GHC). 
STG uses flat environments that can be allocated on the stack, the heap,
or some combination ~\cite{jonesstg}.  

Two other influential lazy evaluation machines relevant to the $\mathcal{CE}$
machine are the call-by-need Krivine machines
~\cite{lkm,krivine2007call,sestoft}, and the three-instruction machine (TIM)
~\cite{TIM}.  Krivine machines started as an approach to call-by-name
evaluation, and were later extended to call-by-need
~\cite{krivine2007call,sestoft,danvy2013synthetic,lkm}.  The $\mathcal{CE}$
modifies the lazy Krivine machine to capture the environment sharing given by
the cactus environment. The TIM is an implementation of call-by-need and
call-by-name ~\cite{TIM}.  It involves, as the name suggests, three machine
instructions, \texttt{TAKE}, \texttt{PUSH}, and \texttt{ENTER}. In
Section~\ref{sec:impl}, we follow Sestoft ~\cite{sestoft} and
re-appropriate these instructions for the $\mathcal{CE}$ machine.

There has also been recent interest in \emph{heapless} abstract
machines for lazy evaluation. Danvy et al. ~\cite{danvy2012inter} and
Garcia et al. ~\cite{garcia2009lazy} independently derived similar
machines from the call-by-need lambda calculus
~\cite{ariola1995call}. These are interesting approaches, but it is not yet
clear how these machines could be implemented efficiently.

\section{Call-by-need}

Call-by-need semantics ensures that arguments are only evaluated when needed,
when needed, and memoize the result so that they are only evaluated once. This
is in contrast to the more common call-by-value, which always evaluates
arguments, and call-by-name, which evaluates arguments for every corresponding
variable dereference. Each semantics has advantages and drawbacks, and
increasingly programming languages are incorporating the ability to switch
between them \cite{?}. 

Our source language is Ariola et al.'s call-by-need lambda calculus \cite{?}. It
is an untyped language with the standard lambda calculus for it's syntax: 

  $$ t := x \; | \; \lambda x . t \; | \; t \; t $$

Ariola et al. give a few different versions of their semantics. To ease 
proof burdens, we use the operational semantics, though formalizing the
syntactic account and it's connection to the operational semantics would be
interesting future work. The semantics are given in Figure~\ref{cbn}. 

Unfortunately the operational semantics have a known issue, which was
independently exposed during our formalization. The issue is that upon
evaluation of a dereferenced variable, part of the heap is forgotten. When
adding fresh variables, the semantics requires that they are fresh with respect
only to the \emph{current} heap, but when merging back with the previously
forgotten heap in the Id rule, the semantics can break the invariant that every
address in the domain of the heap is unique, and can therefore perform incorrect
evaluation. For an example of how it can go wrong see Figure~\ref{cbnwrong}, 

To fix this problem, instead of throwing away the unreachable part of the heap
in the Id rule, we keep it around, but separate the reachable heap from the
unreachable explicitly. We then ensure that freshness is enforced with respect
to both the unreachable and reachable parts of the heap, preventing the issue
described above. This fix has the additional benefit that it simplifies the
relationship to machine code, keeping the heap in tact as a global entity, in
contrast to existing fixes to this issue \cite{?}. The fixed semantics are shown
in Figure~\ref{fig:cbnfixed}. The reachable and unreachable heaps are seperated by
the \texttt{\&} character.

\begin{figure}
\lstinputlisting{cbnbroken.v}
\caption{Original, broken call-by-need semantics}
\label{fig:cbnbroken}
\end{figure}

\begin{figure}
\lstinputlisting{cbnfixed.v}
\caption{Fixed call-by-need semantics}
\label{fig:cbnfixed}
\end{figure}

One might be tempted to ignore the ordering of the heap bindings and try and
reason about a monolithic heap directly. The difficulty that arises with this
approach is that for the Id rule we need to retain that the expression being
entered is unchanged before the update: i.e. there are no cycles in the heap. 
When working with a monolithic heap, one needs to reason about this acyclic
property directly. Unfortunately this approach is much more difficult, as it
requires a lemma showing that unreachable heap locations \emph{stay}
unreachable. Using the approach of Ariola et al. of gives us a much simpler to
reason about approximation of reachability. 

An important property in the proof of bisimulation will be \emph{well-formed}
property. We use the definition directly of Ariola et al., namely that every
term in the heap is \emph{closed to the left}, that is, the free variables of a
term in the heap are in the domain of the heap to the left of said term. In
addition, the term being evaluated is closed under the reachable heap, and all
domain variables are unique with respect to both the reachable and unreachable
heaps. We prove the following lemma, that well-formedness is preserved through
evaluation: 

\begin{lstlisting}
Lemma well_formed_step : ∀ c1 c2, well_formed c1 → c1 ⇓ c2 → well_formed c2.
\end{lstlisting}


For simplicity of formalization, we choose to follow the convention of Ariola et
al. that when we bind a new variable to a new term in the heap, we ensure that
that variable is fresh with respect to both the heap and the term we will be
substituting into.  This prevents reasoning directly about garbage collection
or re-use of heap locations. We will return to this issue in the discussion
section, but it is worth keeping in mind throughout the paper that relaxing the
freshness constraint to being fresh with respect to \emph{live} or
\emph{reachable} heap locations should be possible, and is an interesting
opportunity for future work. 

\subsection{Call-by-name}
An important property of call-by-need is that it is an optimized implementation of
the simpler call-by-name semantics: 

\begin{align}
\inference{l \downarrow \lambda x. b \quad \quad b [m / x] \downarrow v}{l \; m \downarrow v}
\end{align}

Formally, this means that any expression that evaluates to a value in
call-by-name will evaluate to an equivalent value in call-by-need. This is a
valuable property to have, because it means we can reason about
\emph{correctness} properties of a source language, e.g. type preservation,
using these simpler evaluation semantics. 

It's worth noting here that there are cases when requiring call-by-need
semantics is detrimental to performance. The basic idea is that there are cases
when storing a value and referencing it later is more expensive than re-creating
it on demand. For example, the function \texttt{mean xs = sum xs / length xs}
when called on a something like \texttt{[1..10000]} will memoize the list in
memory for the call to \texttt{sum}, then traverse it to compute the
\texttt{length}.  If the memoization restriction was lifted to permit any
non-strict semantics, like call-by-name, this operation could be computed more
quickly entirely in machine registers. While we choose to verify compilation of
simpl call-by-need semantics, we'll return to this discussion of relaxed
non-strict semantics in the Discussion section.


\section{Cactus Environment Big Step Semantics}

Of course, when compiling to machine code, we need a mechanical way to
efficiently implement the substitution and memoization as described in
call-by-need semantics. For this we turn to the recently developed Cactus
Environment $\mathcal{CE}$ Machine \cite{?}. 

We define a cactus environment machine state to be well formed if, like the
call-by-name semantics, a closure bound in the heap is closed to the left.
We also require the closure in question to be closed under the reachable heap,
and all domain variables to be unique across both the unreachable and reachables
heaps. Because we are using deBruijn indices, when referring to a fresh heap
location, we don't require freshness with respect to a substitution term,
because there is no substitution. This will be important when relating the two
states. We have the following lemma showing that well-formedness is preserved
through evaluation TODO.

\begin{figure}
\textbf{Syntax}
\begin{align*}
\tag{Term} t &::= i \; | \; \lambda t \; | \; t \; t  \\
\tag{Variable} i &\in \mathbb{N}  \\
\tag{Closure} c &::= t [\rho] \\
\tag{Environment} \rho &::= \bullet \; | \; c \cdot \rho \\
\end{align*}
\textbf{Semantics}
\begin{align*}
\tag{LEval}\inference
{t_1[\rho] {\xrightarrow{* }}_L \lambda t_2[\rho'] }
{t_1 t_3[\rho] \rightarrow_L t_2[t_3[\rho] \cdot \rho'] } 
\end{align*}
\begin{align*}
\tag{LVar} i [c_0 \cdot c_1 \cdot ... c_i \cdot \rho] \rightarrow_L c_i
\end{align*}
\caption{Curien's call-by-name calculus of closures ~\cite{curien1991abstract}}
\label{fig:calcclos}
\end{figure}

\subsection{Bisimulation with Call-by-Need}

Our bisimulation relation between call-by-need and the big step cactus
environment semantics requires a few different properties. First, we require
that both states are well formed. Second, we need a way to relate heap
locations. Because our freshness conditions vary, we cannot without loss of
generality choose a freshness function that operates on heap domains and use
equality for our heap domain relation. Instead, we must generate and retain an
isomorphism between $\mathcal{CE}$ and CBN heap locations. Thirdly, we need a
way to relate terms with free variables from deBruijn indices to standard
indices. For this, we use an logical relation that corresponds to a type of
\emph{locally nameless} representation \cite{?}. Indeed, one can show that the
relation holds iff the two terms can be translated into equal locally nameless
terms, modulo the free variables being related by the heap location isomorphism.  
See the \texttt{eq\_terms} for the definition of this relation.

For our proof that the relation is a bisimulation, \texttt{cbn\_cem\_bisim}, we
proceed by induction on each of the step relations.  For the \texttt{Id} rules
for each semantics, the proof follows from the fact that   



\section{Cactus Environment Small Step Semantics}

This section describes how we define small step semantics using a stack to track
argument closures from the \texttt{App} rule and update markers from the
\texttt{Id} rule. By defining this intermediate semantics, we ease the proof
burden between the big-step semantics and the assembly machine semantics.   

\subsection{Semantics}

The rules for our small step semantics closely map to the rules of of the big
step semantics. We split the Id and App rules into rules that push and pop off
the stack. The \texttt{Id} rule gets split into the \texttt{Update} and
\texttt{Var'} rules, which take an update marker and replace the closure at that
location with the current value, and push an update marker, respectively. 

One major difference is that we have to drop the reachability requirement here.
Because we have no way to \emph{remember} the reachability ordering without
pushing something onto the stack, we must change our heap to be a monolithic
single heap, and then reason about the existence of a partitioning of the heap
to be equivalent to the big step semantics. This is analagous to a kind of
\emph{erasure} of information about reachability. It does raise questions as to
whether or not there is some way to retain this information for a cheap way to
re-allocate memory, but that is beyond the scope of this work.

Our well formed property for the small step semantics is a bit different, due to
the monolithic heap. We don't actually require any well-formedness property for
the stack, though investigating that possibility would make for interesting
future work. That said, the proof of retention of well-formedeness is much
simpler: there is no inductive step.   

\begin{figure}
\begin{lstlisting}
Inductive step' : state → state → Prop :=
  | Upd : ∀ Φ Υ Ψ b e e' c l s, 
  st Ψ (Φ++(l,cl c e')::Υ) (inr l::s) (close (lam b) e) →s 
  st Ψ (Φ++(l,cl (close (lam b) e) e')::Υ) s (close (lam b) e)
  | Var' : ∀ Υ Φ Ψ s v l c e e', 
  clu v e (Φ++(l,cl c e')::Υ) = Some (l,c) → 
  st (Φ++(l,cl c e')::Υ) Ψ s (close (var v) e) →s 
  st Υ (Ψ++Φ++[(l,cl c e')]) (inr l::s) c
  | Abs' : ∀ Υ Φ b e f c s, 
  f ∉ domain (Υ ++ Φ) → 
  st Υ Φ (inl c::s) (close (lam b) e) →s 
  st ((f, cl c e)::Υ) Φ s (close b f)
  | App' : ∀ Υ Φ e s n m, 
  st Υ Φ s (close (app m n) e) →s 
  st Υ Φ (inl (close n e)::s) (close m e)
where " c1 '→s' c2 " := (step' c1 c2).
\end{lstlisting}
\caption{Small Step $\mathcal{CE}$ Semantics}
\end{figure}

\subsection{Relation to Big Step $\mathcal{CE}$ Semantics}

We prove that the cesm implements the big-step semantics and the reflexive
transitive closure of the small-step semantics. The relation itself is generally
uninteresting; the heap structure is essentially the same so we require equality
of the heap and concatenation of the big step heaps, and we require nothing of
the stack. Furthermore, the terms and closures are equivalent. Really, the only
goal of this proof is to show that the stack preserves the computation
structure.  

\begin{lstlisting}
Lemma bigstep_smallstep : ∀ c h v h' s, 
  big.step (big.st c h) (big.st v h') → 
  refl_trans_clos small.step (small.st c s h) (small.st v s h')
\end{lstlisting}

Note that the relation is defined on the reflexive transitive closure of the
small-step relation \emph{for all stacks}. We use the fact that this implies
that the same relation will hold for the empty stack, which is the initial and
final state of the small-step machine, as desired.


\section{Machine Code}

As described in the introduction, here we describe the machine semantics, and
how the bisimulation with the stack machine from the previous section works.
We can then write down the full bisimulation relation by composing the
relations. We thus end up with our final bisimulation proof, namely that the
call-by-need semantics bisimulate the machine semantics. 

\subsection{Machine Semantics}

The machine semantics are what one would expect given the instructions and
machine state. We omit the full semantics, though they are available in the
source Coq files. We have addition and subtraction for the purposes of
comparing, and the ability to check if a number is zero. 

Some of the less obvious semantics: a closure is represented as two machine
words, or \texttt{nat}s in our case. The first is an instruction pointer. The
second is an environment pointer pointing into the heap. Our current closure is
defined by our instruction pointer and environment pointer registers. 

On the stack, we differentiate between update markers and argument closures by
using a zero in place of an instruction pointer, therefore disallowing a zero
instruction pointer, in agreement with modern conventions. We can then check for
zero on the top of the stack, and in the case of This allows for  

\subsection{Bisimulation with Small Step $\mathcal{CE}$ Semantics}

We create our bisimulation on the basic blocks created by the compiler. We
relate the machine and small step states in fairly simple ways. The deBruijn
terms of the small step semantics are all replaced with pointers into
instruction memory, and we require that the mapping preserved compilation
equivalence. 

For execution of instructions, we relate each rule in the small step semantics
to a basic block of code. Note that we've artificially increased the number of
instructions in this case and could trivivally show that a sound optimization
removing all of the unconditional \texttt{jmp} instructions to the next
instruction. 

In the same way substitution is often modeled as a single step, when
implementing the lookup in the machine semantics we must convert our
\texttt{clu} to a an inductive lookup executed by machine instructions. Of
course, this takes a number of instructions proportional to the size of the
deBruijn index. 

Our heap relation is fairly striaghtforward. Each cell of the $\mathcal{CE}$
semantics corresponds with three machine words: for the closure there will be an
instruction pointer and an environment pointer, and then one machine words for
the environment continuation. A cell is equivalent to one of these triplets iff
the instruction pointer points to a basic block that is equivalent to the term,
and the environment pointers are equivalent modulo heap location isomorphism.

Note that we do require that the \texttt{new} instruction returns a block of
machine words. This is in contrast to flat representations, where it needs to
return blocks of variables sizes. This is also a situation in which the
simplicity of the $\mathcal{CE}$ machine is very valuable: because of this
constant sized closures, we don't need to worry about cases in which the
value closure that we update a heap location with has more free variables, and
therefore requires more space, leading to the need for indirections as in the
STG machine \cite{STG}.



\section{Applications}

Now that we've seen how the compiler preserves the semantics of the source
language, we can look at what sorts of tools this gives us for reasoning about
the output of the compiler. The basic idea is as follows: because of the
strength of the bisimulation relation, almost all reasoning about the source
semantics is preserved through to the resulting binary. This includes properties
like runtime, heap usage, stack size, types, and of course, correctness of
the results.

\subsection{Time}

One of the most common concerns for someone writing source code for a compiler
is runtime. The ability of programmers to reason about their code's performance
is hampered by the common philosophy of compilers: use any means necessary, no
matter how hard to reason about at the source level, to make code run fast. This
is a reasonable default; when reasoning about code performance, it is a very
rare case that the programmer \emph{doesn't} want their code to run as fast as
possible. 

That said, there are cases when the compiler isn't clever enough. The long-lived
hope for the \emph{sufficiently clever compiler} is dead for now. Instead,
programmers are forced to rewrite their code to improve performance. Re-writing
code to improve performance will often require reasoning about opaque decisions
made by the optimizing compiler in question; not an easy task by any measure.
Stream fusion, and other heuristics, are examples of the importance, but also
the fragility of this approach.

\subsection{Types}

When we discuss reasoning about our programs at the source level, it would be
irresponsible to give types anything less than their own subsection. Types are
how we do formal reasoning about programs in practice, and when building a
compiler, we'd like to make sure this reasoning is preserved. Chlipala showed
elegantly how types can be provably preserved through compilation simply typed
strict lambda calculus in \cite{?}. While this paper has focused on untyped
lambda calculus, we use this section to show how the strength of the
bisimulation proof ensures that any type-based reasoning will be preserved
through evaluation.  

Unfortunately, because we've already fixed our term syntax to be untyped, we
cannot do this reasoning within the common context of explicitly typed
langauges, e.g. System F or the Calculus of Constructions. Instead, we must turn
to a completely implicit type system, which makes type judgements on simple
lambda terms, not unlike a subset of existing Hindley-Milner style languages. We
see no reasoning that this line of reasoning should not extend to more powerful
explicit type systems. 

To formalize this, we generalize the notion of a type judgement to any relation
between a source term and some arbtrary type, represented as a Type variable in
Coq. Leaving the type system completely abstract, we define a typing judgement
to be of the form:

\begin{lstlisting}
Variable type : Type.
Variable hasType : relation tm type.
\end{lstlisting}

To reason about our abstract type we turn to a property that just about every
type system must ahere to: preservation. Preservation says that if a term has a
type, and takes one or more steps, its type must be preserved. We choose to
define preservation using the semantics of call-by-name, as they are our
simplest semantics when to use when we aren't concerned with operational
subtleties.

\begin{lstlisting}
Variable preservation : ∀ (t t':tm) (tau:type), 
  hasType t tau → cbname.step t t' →
  hasType t' tau.
\end{lstlisting}

Given a proof of presevation for our type system for call-by-name, we can prove
that the compiled code respects this, similar to Chlipala's proof that his
simply typed lambda calculus compiler was type-preserving.


                                                

\section{Discussion}
In this secion we discuss some of the implications of this work, what the strengths
and weaknesses are, and potential future work. 

Using the natural semantics of the source language to reason about cost is an area
we are excited about extending this work. For example, incorportating substructural
type systems that help reason about memory consumption could enable easier provably
correct programs that are guaranteed to succeed on resource-constrained hardware. This
could allow easier development of critical applications, easing the burden of proof as 
well as development time. By incorporating an appropriate type system into the
proof infrastructure given in this paper, one can imagine a system where a
program that type-checks comes implicitly with a proof that the compiled
program \emph{truly cannot go wrong}, i.e. it cannot run out of time or memory. 

Another straightforward extension to this work would be extending it to implement the
Haskell language. As we discussed earlier in the paper, Haskell programmers
often implicitly rely on call-by-need semantics, so defining a compiler option
that provably guarantees such a semantics could be valuable for many programmers. This
would require more resources than those available to these authors, but it could make
for a useful extension to the DeepSpec project \cite{}. It would also fill a natural hole
in the language space, as existing call-by-value functional languages have a verified 
compiler to use: CakeML, while those who prefer laziness by default currently have no
options for verified compilation. It's worth noting that a more complete
compiler to native code for this abstract machine does exist \cite{cem}. This
would ease the burden of such a project. 

\subsection{Threats to Validity} 
There are a few threats to the validity of this paper that we wish to address
here. The first, and strongest, is that because this is of course a
turing-complete source language, we'd like to have the correctness theorem be
an if and only if. That is, we'd like to say that the resulting machine code
computes a value if and only if the source semantics do. Currently, the
correctness theorem allows cases where the source semantics doesn't terminate,
but the machine semantics do. Our only defense to this argument is that this is 
a common challenge for proving correctness, and proving the implication the
other direction is quite difficult. Existing work similarly avoids this issue, e.g. 
\cite{chlipala}. Indeed, this is a direct result of the difficulty of reasoning
about small-step semantics of programs without the nice induction rules given
by natural semantics. Note that this is not an issue in the presence of total
languages, where the source semantics will always terminate.

A similar potential weakness of this paper is the choice of simple assembly
language.  In particular, the use of infinite stack and heap sizes could be
seen as problematic, as no machine that we're aware of has such a large memory
capacity. Proving correctness in presence of limited stack and heap sizes would
require adding the caveats that correctness won't be preserved in the presence
of out-of-memory errors. We argue that our cost model helps to reason about
these issues at the big-step source semantics, one of the contributions of this
paper, so that heuristic runtime checks for potential out of memory errors can
be reasoned about more easily. Valuable future work would include lowering to
one or more verified ISA's, such as those described in \cite{compcert}. 

Another thing that we chose not to do is relate the source semantics directly to an
existing call-by-need semantics. A couple of obvious choices would have been
Launchbury's call-by-need semantics \cite{launchbury} or the operational
call-by-need semantics given by Ariola et al. in \cite{ariola}. While we believe this
would be worthwhile future work, we decided against this for two reasons. The
first, more important one, is that both of these semantics have been shown to
be problematic upon further inspection. Brietner showed that upon machine formalization, 
Launchbury's semantics require some small changes, while we re-discovered an issue with
Ariola et al.'s call-by-need semantics involving freshness that complicates the
semantics significantly, previously discovered by \cite{?}. While we could have used
corrected versions of either, in both cases the semantics are complicated by
their fixes, and we found relating our semantics to the more "obviously
correct" call-by-name semantics of Curien more satisfying. The second reason is simply
that relating two different semantics operating on two different calculi
(standard and deBruin indices) with mutable heaps via bisimulation turns out to
be quite challenging. Again, this relation, and formaling the relation to the
syntactic account given by Ariola et al. would make for exciting future work.




\section{Conclusion}

We have shown how to build a verified compiler that verifies time and space
requirements in addition to the usual correctness properties. We have used this
technique to build a verified compiler of call-by-need, proving that the
memoization of results is implemented correctly. This is the first
machine-checked proof that an optimization is \emph{preserved through
compilation} that we are aware of. While we haven't proved that the memoization
of results is a true optimization, and it certainly isn't in general, we have
built a framework where that sort of reasoning is possible.

Typed functional language programmers have long leaned on the important property
that their programs "can't go wrong". Historically this property must always
include an important asterisk: "unless it runs out of memory". Without tools to
reason formally about memory usage this asterisk is impossible to confront. We
have shown how one can reason formally about properties like stack size and heap
size, in a way that is preserved to machine code, enabling programmers to
finally manage this wart.



\section{Acknowledgments}
Sandia is a multiprogram laboratory operated by Sandia Corporation, a Lockheed Martin Company, for the United States Department of Energy’s National Nuclear Security Administration under contract DE-AC04-94AL85000.

% We recommend abbrvnat bibliography style.
\bibliographystyle{abbrvnat}
\bibliography{annotated}

\end{document}
