\section{Cactus Environment Small Step Semantics}

This section describes how we define small step semantics using a stack to track
argument closures from the \texttt{App} rule and update markers from the
\texttt{Id} rule. By defining this intermediate semantics, we ease the proof
burden between the big-step semantics and the assembly machine semantics.   

\subsection{Semantics}

The rules for our small step semantics closely map to the rules of of the big
step semantics. We split the Id and App rules into rules that push and pop off
the stack. The \texttt{Id} rule gets split into the \texttt{Update} and
\texttt{Var'} rules, which take an update marker and replace the closure at that
location with the current value, and push an update marker, respectively. 

One major difference is that we have to drop the reachability requirement here.
Because we have no way to \emph{remember} the reachability ordering without
pushing something onto the stack, we must change our heap to be a monolithic
single heap, and then reason about the existence of a partitioning of the heap
to be equivalent to the big step semantics. This is analagous to a kind of
\emph{erasure} of information about reachability. It does raise questions as to
whether or not there is some way to retain this information for a cheap way to
re-allocate memory, but that is beyond the scope of this work.

Our well formed property for the small step semantics is a bit different, due to
the monolithic heap. We don't actually require any well-formedness property for
the stack, though investigating that possibility would make for interesting
future work. That said, the proof of retention of well-formedeness is much
simpler: there is no inductive step.   

\begin{figure}
\begin{lstlisting}
Inductive step' : state → state → Prop :=
  | Upd : ∀ Φ Υ Ψ b e e' c l s, 
  st Ψ (Φ++(l,cl c e')::Υ) (inr l::s) (close (lam b) e) →s 
  st Ψ (Φ++(l,cl (close (lam b) e) e')::Υ) s (close (lam b) e)
  | Var' : ∀ Υ Φ Ψ s v l c e e', 
  clu v e (Φ++(l,cl c e')::Υ) = Some (l,c) → 
  st (Φ++(l,cl c e')::Υ) Ψ s (close (var v) e) →s 
  st Υ (Ψ++Φ++[(l,cl c e')]) (inr l::s) c
  | Abs' : ∀ Υ Φ b e f c s, 
  f ∉ domain (Υ ++ Φ) → 
  st Υ Φ (inl c::s) (close (lam b) e) →s 
  st ((f, cl c e)::Υ) Φ s (close b f)
  | App' : ∀ Υ Φ e s n m, 
  st Υ Φ s (close (app m n) e) →s 
  st Υ Φ (inl (close n e)::s) (close m e)
where " c1 '→s' c2 " := (step' c1 c2).
\end{lstlisting}
\caption{Small Step $\mathcal{CE}$ Semantics}
\end{figure}

\subsection{Relation to Big Step $\mathcal{CE}$ Semantics}

We prove that the cesm implements the big-step semantics and the reflexive
transitive closure of the small-step semantics. The relation itself is generally
uninteresting; the heap structure is essentially the same so we require equality
of the heap and concatenation of the big step heaps, and we require nothing of
the stack. Furthermore, the terms and closures are equivalent. Really, the only
goal of this proof is to show that the stack preserves the computation
structure.  

\begin{lstlisting}
Lemma bigstep_smallstep : ∀ c h v h' s, 
  big.step (big.st c h) (big.st v h') → 
  refl_trans_clos small.step (small.st c s h) (small.st v s h')
\end{lstlisting}

Note that the relation is defined on the reflexive transitive closure of the
small-step relation \emph{for all stacks}. We use the fact that this implies
that the same relation will hold for the empty stack, which is the initial and
final state of the small-step machine, as desired.

